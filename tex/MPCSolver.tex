\documentclass[]{article}

\usepackage{amsthm} %qed
\usepackage[cmex10]{amsmath}
\usepackage{amsfonts}
\usepackage{amssymb}
\usepackage{amsmath}
\usepackage{graphicx}
\usepackage{color}


\newcommand{\BIN}{\begin{bmatrix}}
\newcommand{\BOUT}{\end{bmatrix}}
\newcommand{\cred}[1]{\textcolor{red}{#1}}
\newcommand{\cblue}[1]{\textcolor{blue}{#1}}
\newcommand\eqbydef{\mathrel{\overset{\makebox[0pt]{\mbox{\normalfont\tiny\sffamily def}}}{=}}}

%\newcommand{\mdiff1}[1]{\frac{\partial}{\partial #1}}
\newcommand{\diff}[2]{\dfrac{\partial #1}{\partial #2}}
\newcommand{\norm}[1]{\left\| #1 \right\|}
\newcommand{\SC}[1]{\begin{flushleft}{\color{red} SC: #1}\end{flushleft}}

\DeclareMathOperator*{\minimize}{\min.}
\newcommand{\st}{\mbox{s.t.}}

\setcounter{MaxMatrixCols}{20}


\begin{document}

\title{\Large Dedicated optimization for MPC with convex boundedness constraint}
\author{Adrien Escande}

\maketitle

\section{Overview -- main idea}
We are considering a problem of the form
\begin{align}
  \minimize_{\Phi \in \mathbb{R}^n}\ & \left\| J \Phi \right\|^2 \label{eq:problem}\\
  \st \ & \sum_{j=0}^{n-1} \frac{a_j}{\sqrt{\Phi_{j+1}} + \sqrt{\Phi}_j} - \alpha \sqrt{\Phi_n} = b \nonumber\\
        & l_j \leq \Phi_j - \Phi_{j-1} \leq u_j, \quad j=1..n \nonumber\\
        & \Phi_n^- \leq \Phi_n \leq \Phi_n^+ \nonumber
\end{align}
where $\alpha$ is either $0$ or positive and, by convention, $\Phi_0 = 0$. Lower and upper bounds can be equal to denote an equality constraint.

The idea we want to explore here is to pass the equality constraint in the objective as a weighted squared norm. Denoting $f$ the function such that the non-linear constraint writes $f(\Phi) = 0$, we thus transform the above problem into
\begin{align}
  \minimize_{\Phi \in \mathbb{R}^n}\ & \left\| J \Phi \right\|^2 + w^2 \left\|f\right\|^2\\
  \st \ 
    % & \sum_{j=0}^{n-1} \frac{a_j}{\sqrt{\Phi_{j+1}} + \sqrt{\Phi}_j} - \alpha \sqrt{\Phi_n} = b \nonumber\\
        & l_j \leq \Phi_j - \Phi_{j-1} \leq u_j, \quad j=1..n \nonumber\\
        & \Phi_n^- \leq \Phi_n \leq \Phi_n^+ \nonumber
\end{align}
\SC{commented the nonlinear equality constraint}
This is an inexact method, but by paying attention to the order of operations
we can have $w$ high. The above problem is a non-linear least square
optimization with linear constraints. It can be solved with an SQP. The idea is
that the underlying QP is a linear least square that we can solve much faster
than usual by taking advantage of the form of the matrix $J$ and of the
constraints.

\section{Linear constraints}
We focus here on the constraints
\begin{align*}
 & l_j \leq x_j - x_{j-1} \leq u_j, \quad j=1..n \label{eq:inequality}\\
 & x_n^- \leq x_n \leq x_n^+
\end{align*}

\subsection{Constraint matrix}
The first set of inequality constraints can be written $l \leq C_Z \leq u$ with
\begin{equation}
  C_Z = \BIN  1 &  0 & 0 & \ldots &  0 & 0\\
             -1 &  1 & 0 & \ldots &  0 & 0\\
              0 & -1 & 1 &        &  0 & 0\\
                & \vdots &&\ddots &    & \vdots \\
              0 &  0 & 0 & \ldots & -1 & 1\BOUT 
\end{equation}
This matrix has full rank.

The whole linear constraints writes
\begin{equation}
  \BIN l \\ x_n^- \BOUT \leq C x \leq \BIN u \\ x_n^+ \BOUT, \quad \mbox{with}\ C = \BIN C_Z \\ e_n^T\BOUT
  \label{eq:feasibility}
\end{equation}
where $e_n$ is the $n$-th column of the $n\times n$ identity matrix.

\subsection{Expression as a zonope}
Let's define $H = \left\{x \in \mathbb{R}^n \vert\ l \leq C_Z \leq u\right\}$, 
\begin{equation}
L = \BIN 1 & 0 & \ldots & 0 \\ 1 & 1 &\ldots & 0 \\ & & \ddots & \\ 1 & 1 &\ldots & 1\BOUT, \qquad D = \BIN u_1 - l_1 & & 0 \\ & \ddots & \\ 0 & & u_n - l_n\BOUT
\end{equation}
and $Z = Ll + LD \left[0,1\right]^n$. Z is a zonotope. We show that $H = Z$.\newline
TODO: proof
\newline
The constraints~(\ref{eq:feasibility}) are thus a zonotope cut by zero to two ``horizontal'' hyperplanes.
Note that, alternatively, $Z = Ll + L \mathcal{B}$ where $\mathcal{B}$ is the box defined by $l$ and $u$.

\subsection{Nullspace of active constraints}
Let's $\mathcal{A}$ be an active set and define $C_\mathcal{A}$ as the submatrix of $C$ corresponding to the active constraints.
Because $C_Z$ is full rank and $e_n^T = \BIN 1 & 1 & \ldots & 1 \BOUT C_Z$ (\emph{i.e.} is a linear combination of \emph{all} lines of $C_Z$), $C_\mathcal{A}$ is always full row rank, except for the obvious case $\mathcal{A} = 1..n+1$ where $C_\mathcal{A}$ is $n+1$ by $n$. This case does not arise in practice.

$\mathcal{A}$ can be described in the following way: starting at the first constraint, count the number $a_0$ of consecutive active constraints (possibly $0$ if the first constraint is not active), then $i_1$ the number of consecutive inactive constraints following the $a_0$ first constraints, $a_1$ the following number of active constraints,~\ldots. $\mathcal{A}$ is then fully described by the sequence $(a_0, i_1, a_1, i_2, a_2, \ldots, i_p, a_p)$, where all $a_k$ and $i_k$ are at least $1$, except for $a_0$ and $a_p$, that can be $0$. We have that $\sum a_k + \sum i_k = n+1$, and we define $n_{\mathcal{A}} = \sum a_k$.
For example, if $\mathcal{A}_{\mbox{\small ex}} = \left\{1, 2, 6, 7, 8, 10\right\}$, with $n=10$, we get the sequence $(2,3,3,1,1,1,0)$, and $n_{\mathcal{A}} = 6$.

$C_\mathcal{A}$ is a $n_{\mathcal{A}}\times n$ matrix:
\begin{equation}
  C_\mathcal{A} = 
  \BIN C_0 &     &     &     &     &        &     &    \\
           & 0_{a_1, i_1-1} & C_1 &     &     &        &     &    \\
           &     &     & 0_{a_2, i_2-1} & C_2 &        &     &    \\
           &     &     &     &     & \ddots &     &    \\
           &     &     &     &     &        & 0_{a_p, i_p-1} & C_p\BOUT
\end{equation}
with $C_0$, $C_k$ ($k=1..p-1$) and $C_p$ respectively $a_0\times a_0$, $a_k\times (a_k+1)$ and $a_p\times a_p$ matrices with form
\begin{equation}
  C_0 = \BIN 1 &        &        &   \\
            -1 &    1   &        &   \\
               & \ddots & \ddots &   \\
               &        &   -1   & 1 \BOUT, \
  C_k = \BIN-1 &    1   &        &   \\
               & \ddots & \ddots &   \\
               &        &   -1   & 1 \BOUT, \
  C_p = \BIN-1 &    1    &        &   \\
               & \ddots  & \ddots &   \\
               &         &   -1   & 1 \\
               &         &        & 1 \BOUT \nonumber
\end{equation}
(note that $C_0$ and $C_p$ can be empty), and the $0_{a,i}$ are $a\times i$ zeros matrices. Continuing the example, we have (with $C_k$'s in red and $0_k$'s in blue)
\begin{equation}
%C_{\mathcal{A}_{\mbox{\small ex}}} = \BIN                                           
   %1 & 0 &   &   &    &    &    &   &    &   \\
  %-1 & 1 &   &   &    &    &    &   &    &   \\
     %&   & 0 & 0 & -1 &  1 &  0 & 0 &    &   \\
     %&   & 0 & 0 &  0 & -1 &  1 & 0 &    &   \\
     %&   & 0 & 0 &  0 &  0 & -1 & 1 &    &   \\
     %&   &   &   &    &    &    &   & -1 & 1 \\
%\BOUT
C_{\mathcal{A}_{\mbox{\small ex}}} = \BIN                                           
   \cred{1} & \cred{0} & 0 & 0 &  0 &  0 &  0 & 0 &  0 & 0 \\
  \cred{-1} & \cred{1} & 0 & 0 &  0 &  0 &  0 & 0 &  0 & 0 \\
   0 & 0 & \cblue{0} & \cblue{0} & \cred{-1} & \cred{1} & \cred{0} & \cred{0} &  0 & 0 \\
   0 & 0 & \cblue{0} & \cblue{0} & \cred{0} & \cred{-1} & \cred{1} & \cred{0} &  0 & 0 \\
   0 & 0 & \cblue{0} & \cblue{0} & \cred{0} & \cred{0} & \cred{-1} & \cred{1} &  0 & 0 \\
   0 & 0 & 0 & 0 &  0 &  0 &  0 & 0 & \cred{-1} & \cred{1} \\
\BOUT
\end{equation}
We define $1_a$ as the vector of size $a$ filled with ones and
\begin{equation}
  N_{\mathcal{A}} = \BIN 
    0_{a_0,i_1-1} & & & & & \\
    I_{i_1-1} & & & & & \\
    & 1_{a_1+1} & & & & \\
    && I_{i_2-1} & & & \\
    &&& 1_{a_2+1} & & \\
    &&&&\ddots & \\
    &&&&& I_{i_p-1} \\
    &&&&& 0_{a_p,i_p-1}
  \BOUT
\end{equation}
Noting that $C_k 1_{a_k+1} = 0_{a_k,1}$, we can directly verify that $C_{\mathcal{A}} N_{\mathcal{A}} = 0$. $N_{\mathcal{A}}$ is $n$ by $n-n_{\mathcal{A}}$ and full column rank. It is thus a basis of the nullspace of $C_{\mathcal{A}}$. It can even be made an orthonormal basis by normalizing the columns, \emph{i.e.} replacing $1_{a_k+1}$ by $\dfrac{1}{\sqrt{a_k+1}}1_{a_k+1}$.

With our example,
\begin{equation}
N_{\mathcal{A}_{\mbox{\small ex}}} = \BIN              
  \cred{0} & \cred{0} & 0 & 0 \\
  \cred{0} & \cred{0} & 0 & 0 \\
  \cblue{1} & \cblue{0} & 0 & 0 \\
  \cblue{0} & \cblue{1} & 0 & 0 \\
  0 & 0 & \cred{1} & 0 \\
  0 & 0 & \cred{1} & 0 \\
  0 & 0 & \cred{1} & 0 \\
  0 & 0 & \cred{1} & 0 \\
  0 & 0 & 0 & \cred{1} \\
  0 & 0 & 0 & \cred{1} \\
\BOUT
\end{equation}
Pre- or post-multiplication by $N_{\mathcal{A}}$ can be done very efficiently without ever forming $C_{\mathcal{A}}$ or $N_{\mathcal{A}}$, or actually perform a multiplication.

\subsection{Pseudoinverse of $C_{\mathcal{A}}$}
Because of the block structure of $C_{\mathcal{A}}$, obtaining the pseudoinverse is a matter of finding the pseudoinverse for each $C_k$. Indeed
\begin{equation}
  C_{\mathcal{A}}^+ = \BIN
    C_0^{-1} & & & & \\
    & 0_{i_1-1,a_1} & & & \\
    & C_1^+& & &\\
    && 0_{i_2-1,a_2} & & & \\
    && C_2^+&  & &\\
    &&& \ddots & \\
    &&&&0_{i_p-1,a_p} \\
    &&&& C_p^{-1}
  \BOUT
\end{equation}
We have
\begin{equation}
  C_0^{-1} = \BIN 
    1 &   & \\
    \vdots & \ddots & \\
    1 & \ldots & 1 \BOUT, \quad 
  C_p^{-1} = \BIN 
    -1 & \ldots & -1 & 1 \\
       & \ddots & \vdots & 1\\
       &        &  -1   & 1 \\
       &&& 1 \BOUT
\end{equation}
and
\begin{equation}
 C_k^+ = \frac{1}{a_k+1} \BIN
   -a_k & -(a_k-1) & -(a_k-2) & \ldots & -1 \\
     1  & -(a_k-1) & -(a_k-2) & \ldots & -1 \\
     1  &    2     & -(a_k-2) & \ldots & -1 \\
     1  &    2     &    3     & \ddots & \vdots \\
        & \vdots   &          & \ddots & -1 \\
     1  &    2     &    3     & \ldots & a_k
 \BOUT
\end{equation}
Again, the multiplication by the pseudoinverse can be done very efficiently.


\section{Feasibility}
\subsection{Linear feasibility and initial point}
The points verifying the linear constraints are $x = Ll + \sum \lambda_i (u_i-l_i) L_i, \ \lambda \in \left[0,1\right]^n$ (that is $x \in H$), such that $x_n^- \leq x_n \leq x_n^+$, where the $L_i$ are the columns of $L$.\newline
In $H$, we have $x_n = \sum l_i + \sum \lambda_i (u_i-l_i)$. The minimum and maximum values are obtained exclusively for $\lambda = 0$ and $\lambda = 1_n$.

Let's define $x(\alpha) = Ll + \sum \alpha (u_i-l_i) L_i$. For $\alpha \in \left[0,1\right]$, $x(\alpha) \in H$ and the minimum and maximum values of $x_n$ in $H$ are attained for $\alpha = 0$ and $\alpha = 1$, respectively. $x_n(\alpha)$ is a linear, increasing function of $\alpha$: $x_n(\alpha) = s_l + \alpha s_d$ where $s_l = \sum l_i$ and $s_d = \sum u_i-l_i$. \newline
We can compute the values of $\alpha^-$ and $\alpha^+$ such that $x_n(\alpha^-) = x_n^-$ and $x_n(\alpha^+) = x_n^+$:
\begin{align}
 \alpha^- = \frac{x_n^- - s_l}{s_d} \\
 \alpha^+ = \frac{x_n^+ - s_l}{s_d}
\end{align}
The linear constraints are feasible if the intersection $\left[\alpha^-, \alpha^+\right] \cap \left[0,1\right]$ is not empty. In this case, any $\alpha$ in this intersection yields a feasible point $x(\alpha)$ for the linear constraints. We can chose the middle point $\alpha_m = \frac{\max(\alpha^-,0) + \min(\alpha^+,1)}{2}$.

\subsection{Feasibility for all constraints}
One way to check the feasibility of the problem~(\ref{eq:problem}) is to solve the following:
\begin{align}
  \minimize_x\ & \left\|c(x)\right\|^2 \\
  \st\ & l \leq C_Z x \leq u\\
       & x_n^- \leq x_n \leq x_n^+
\end{align}
where
\begin{equation}
  c(x) = \sum_{j=0}^{n-1} \frac{a_j}{\sqrt{x_{j+1}} + \sqrt{x_j}} - \alpha \sqrt{x_n} = b
\end{equation}
This can be done by solving a SQP with a Gauss-Newton approximation of the objective's Hessian matrix. If at the solution $c(x^*) = 0$, the problem is feasible and $x^*$ is a feasible point.

The SQP in itself does not deserve a particular treatment: it can be a simple adaptation of the line-search SQP algorithm found in~\cite{nocedal:book:2006} (Algorithm 18.3), with the initial point taken as described in the previous section (and initial Lagrange multiplier thus set to $0$). Since all constraints are linear, the merit function is simply our objective function and there is no need for a penalty parameter. The only refinement is the use of double sided constraints, so that the KKT conditions checking has to be modified (see~\cite{brossette:phd:2016}, section 4.3.5).

The underlying QP is where the computational savings are done. It is actually a a constrained least square (written at iterate $x^k$, with the usual shortcuts $c_k = c(x_k)$ and $g_k = \nabla_{x}c(x_k)$):
\begin{align}
  \minimize_p \ & \left\|g_k^T p + c_k\right\|^2 \\
  \st \ & l-C_Z x^k \leq C_Z p \leq u-C_Z x^k \\
        & x_n^- - x^k_n \leq p_n \leq x_n^+ - x^k_n
\end{align}

At given iteration of this QP, let the active set be $\mathcal{A}$. The corresponding EQP is
\begin{align}
  \minimize_p \ & \left\|g_k^T (p+d) + c_k\right\|^2 \\
  \st \ & C_{\mathcal{A}} d = 0
\end{align}
$d$ can be written $d = N_{\mathcal{A}} z$ and the EQP can be transformed simply into $\minimize_z \left\|g_k^T N_{\mathcal{A}}z + c_k + g_k^T p\right\|^2$ whose least norm solution is $z = - (g_k^T N_{\mathcal{A}})^+ (c_k + g_k^T p) = - \dfrac{N_{\mathcal{A}}^T g_k}{\left\|N_{\mathcal{A}}^T g_k\right\|^2}(c_k + g_k^T p)$. Of course, one should make take advantage of the basis $N_{\mathcal{A}}$ described in the previous section.

\emph{Remark}: it is not mandatory to take the least-norm solution here, and one may question this choice when we don't take an orthogonal basis $N_{\mathcal{A}}$, but the pseudo-inverse in this particular case is very cheap to compute and avoid having to consider numerical issues in the choice of the solution.

When needed, the Lagrange multipliers are obtained by solving
\begin{equation}
  C_{\mathcal{A}}^T \lambda_{\mathcal{A}} = (g_k^T p +c) g_k
\end{equation}
for the multipliers corresponding to active constraints, and taking $\lambda_{\bar{\mathcal{A}}} = 0$ for the others. We can take $\lambda_{\mathcal{A}} = (g_k^T p+c) C_{\mathcal{A}}^{+T}g_k$, using the pseudo inverse of $C_{\mathcal{A}}$ as described in the previous section.


\section{Leveraging the cost function form}
\subsection{Matrix form}
We will now consider the cost function
\begin{equation}
  f(x) = \sum_{i=1}^{n-1} \left(d_j(x_{j+1} - x_j) + d_{j-1}(x_j - x{j-1})\right)^2
\end{equation}
with $x_0 = 0$, where $d_i>0$.
It can be rewritten
\begin{align*}
f(x) &= \left\|\BIN -(d_0+d_1)x_1 + d_1 x_2 \\ 
                    d_1 x_1 -(d_1+d_2) x_2 + d_2 x_3 \\ 
                    d_2 x_2 -(d_2+d_3) x_3 + d_3 x_4 \\ 
                    \vdots \\ 
                    d_{n-2} x_{n-2} -(d_{n-2}+d_{n-1}) x_{n-1} + d_{n-1} x_{n-2} \BOUT\right\|^2 \\
 &= \left\|\BIN -d_0-d_1 \hspace{-3pt}& d_1 & & & &\\ 
                d_1 &\hspace{-3pt} -d_1-d_2 \hspace{-3pt}& d_2 & & &\\ 
                & d_2 & \hspace{-3pt}-d_2-d_3 \hspace{-3pt}& d_3 & & \\ 
                & & & \ddots &  & \\ 
                &&& d_{n-2} & -d_{n-2}-d_{n-1} & d_{n-1} \BOUT
 \BIN x_1 \\ x_2 \\ x_3 \\ \vdots \\x_n \BOUT\right\|^2 \\
 & \eqbydef \left\|J x\right\|^2
\end{align*}
Matrix $J$ has several interesting properties:
\begin{enumerate}
	\item it is tridiagonal
  \item it is upper Hessenberg (upper triangular + sub-diagonal)
  \item summing the $3$ non-zero elements of a row or a column gives $0$ (when there are 3 such elements).
  \item it is full row rank
  \item $J(1:n-1,1:n-1)$ is symmetric.
\end{enumerate}

\subsection{QR decomposition}
Because $J$ is tridiagonal, we can very efficiently obtain its QR decomposition (in $O(n)$ whereas for a dense matrix it is $O(n^3)$). To do so use the Hessenberg QR algorithm (see Algorithm 5.2.3 in~\cite{golub:book:1996}) adapted to tridiagonal matrix. We get
\begin{equation}
  J = Q R
\end{equation}
where $Q$ is an $n-1$ by $n-1$ orthogonal matrix and $R$ is an $n-1$ by $n$ upper triangular matrix. $R$ is actual a band matrix with lower bandwidth $0$ and upper bandwidth $2$. We do not explicitly compute $Q$ but keep it as the sequence of Givens rotations used in the decomposition. \newline
\emph{Remark}: In our practical cases, the matrix $J$ is well conditioned. It is thus justified to use a non-pivoting scheme.

Later, we will need the QR decomposition of $\BIN g^T \\ J \BOUT$. Given the above decomposition of $J$, we have
\begin{equation}
  \BIN g^T \\ J \BOUT = \BIN I & 0 \\ 0 & Q \BOUT \BIN g^T \\ R \BOUT
\end{equation}
where the last matrix is upper Hessenberg. We can thus perform cheaply its QR decomposition
\begin{equation}
  \BIN g^T \\ J \BOUT = \BIN I & 0 \\ 0 & Q \BOUT \BIN Q_{g,1} \\ Q_{g,2} \BOUT R_g = \BIN Q_{g,1} \\ Q Q_{g,2} \BOUT R_g
\end{equation}
It might happen that $g^T$ is linearly dependent in some lines of $J$. This is not a problem for the decomposition, and we just need to check the last element of the diagonal to detect this case.


\subsection{Projection on the on $\mathcal{N}(C_\mathcal{A})$}
Let's consider a slightly refined description of $N_\mathcal{A}$, where we split the vectors $1_{a_i+1}$ into $\BIN 1 & 1_{a_i-1}^T & 1\BOUT$ (the central part can be empty if $a_i = 1$):
\begin{equation}
  N_{\mathcal{A}} = \BIN 
    0_{a_0,i_1-1} & & & & & & & \\
    I_{i_1-1} & & & & & & &\\
    & 1 & & & & & &\\
    & 1_{a_1-1} & & & & & &\\
    & 1 & & & & & &\\
    && I_{i_2-1} & & & & &\\
    &&& 1 & & & &\\
    &&& 1_{a_2-1} & & & & \\
    &&& 1 & & & &\\
    &&&& I_{i_3-1} & & & \\
    &&&&&\ddots & & \\
    &&&&&& 1_{a_{p-1}-1} & \\
    &&&&&& 1 & \\
    &&&&&&& I_{i_p-1} \\
    &&&&&&& 0_{a_p,i_p-1}
  \BOUT
\end{equation}
We write $J$ conformably for multiplication (we study first the begining) as
\begin{align*}
  &J =\\ &\BIN
    D_0   & B_0    &          &        &           &        &          &        &           &             &                &            &           \\
    B_0^T & D_1    & b_1'     &        &           &        &          &        &           &             &                &            &           \\
          & b_1'^T & \delta_2 & b_2^T  &           &        &          &        &           &             &                &            &           \\
          &        & b_2      & D_2    & b_2'      &        &          &        &           &             &                &            &           \\
          &        &          & b_2'^T & \delta_2' & b_3^T  &          &        &           &             &                &            &           \\
          &        &          &        & b_3       & D_3    & b_3'     &        &           &             &                &            &           \\
          &        &          &        &           & b_3'^T & \delta_4 & b_4^T  &           &             &                &            &           \\
          &        &          &        &           &        & b_4      & D_4    & b_4'      &             &                &            &           \\
          &        &          &        &           &        &          & b_4'^T & \delta_4' & b_5^T       &                &            &           \\
          &        &          &        &           &        &          &        &           & \ddots      &                &            &           \\ 
          %&        &          &        &           &        &          &        &           & b_{2p-2}'^T & \delta_{2p-2}' & b_{2p-1}^T &           \\
          %&        &          &        &           &        &          &        &           &             & b_{2p-1}       & D_{2p-1}   & \BIN B_{2p} \hspace{-3pt}&\hspace{-2pt} 0 \BOUT \\
          %&        &          &        &           &        &          &        &           &             &                & B_{2p}^T   & D_{2p}    \\
  \BOUT
\end{align*}
where the $\delta$s are scalar, the $D_i$s are square, tridiagonal, symmetric matrices, $B_0$ is a matrix with all elements to 0 but the bottom left one, the $d_i$s are vector whose only non zero element is the first, and the $d_i'$s are vectors whose only non zero element is the last.
We have
\begin{align*}
&J_\mathcal{A} \eqbydef JN_{\mathcal{A}} = \\
&\BIN
    B_0    &                      &        &                      &        &                                &            \\
    D_1    & b_1'                 &        &                      &        &                                &            \\
    b_1'^T & \delta_2 + b_2^T 1   &        &                      &        &                                &            \\
           & b_2 + D_2 1 + b_2'   &        &                      &        &                                &            \\
           & b_2'^T 1 + \delta_2' & b_3^T  &                      &        &                                &            \\
           & b_3                  & D_3    & b_3'                 &        &                                &            \\
           &                      & b_3'^T & \delta_4 + b_4^T 1   &        &                                &            \\
           &                      &        & b_4 + D_4 1 + b_4'   &        &                                &            \\
           &                      &        & b_4'^T 1 + \delta_4' & b_5^T  &                                &            \\
           &                      &        &                      & \ddots &                                &            \\
           %&                      &        &                      &        & b_{2p-2}'^T 1 + \delta_{2p-2}' & b_{2p-1}^T \\
           %&                      &        &                      &        & b_{2p-1}                       & D_{2p-1}   \\
           %&                      &        &                      &        &                                & B_{2p}^T   \\
  \BOUT
\end{align*}
where we dropped the size of the vectors $1_a$.

Because the property $3$ of $J$, we have $b_{2i} + D_{2i} 1_{a_i-1} + b_{2i}' = 0$ for $i=1..p-1$:
\begin{align*}
&J_\mathcal{A} = 
\BIN
    B_0    &                      &        &                      &        &                                &            \\
    D_1    & b_1'                 &        &                      &        &                                &            \\
    b_1'^T & \delta_2 + b_2^T 1   &        &                      &        &                                &            \\
           & \cred{0_{a_1-1,1}}   &        &                      &        &                                &            \\
           & b_2'^T 1 + \delta_2' & b_3^T  &                      &        &                                &            \\
           & b_3                  & D_3    & b_3'                 &        &                                &            \\
           &                      & b_3'^T & \delta_4 + b_4^T 1   &        &                                &            \\
           &                      &        & \cred{0_{a_2-1,1}}   &        &                                &            \\
           &                      &        & b_4'^T 1 + \delta_4' & b_5^T  &                                &            \\
           &                      &        &                      & \ddots &                                &            \\
           %&                      &        &                      &        & b_{2p-2}'^T 1 + \delta_{2p-2}' & b_{2p-1}^T \\
           %&                      &        &                      &        & b_{2p-1}                       & D_{2p-1}   \\
           %&                      &        &                      &        &                                & B_{2p}^T   \\
  \BOUT
\end{align*}
With this same property, we also have the fact that $\delta_{2j} + b_{2j}^T 1$ is equal to the opposite of the non-zero element of $b_{2j-1}'$ (and the same for $b_{2j}'^T 1 + \delta_{2j}'$ and $b_{2j+j}$).
We can consider several submatrices:
\begin{align*}
  &J_0 = \BIN
     B_0    &                    \\
     D_1    & b_1'               \\
     b_1'^T & \delta_2 + b_2^T 1
  \BOUT, \
  J_j = \BIN
    b_{2j}'^T 1 + \delta_2' & b_{2j+1}^T  &                              \\
    b_{2j+1}                & D_{2j+1}    & b_{2j+1}'                    \\
                            & b_{2j+1}'^T & \delta_{2j+2} + b_{2j+2}^T 1 
  \BOUT, j=1..p-2\\
  %&J_{p-1} = \BIN
  %   b_{2p-2}'^T 1 + \delta_{2p-2}' & b_{2p-1}^T \\
  %   b_{2p-1}                       & D_{2p-1}   \\
  %                                  & B_{2p}^T   \\
  %\BOUT
\end{align*}
The study of the end of $J_\mathcal{A}$ is a bit more tedious as we need to separate the cases, depending on the values of $a_p$ and $i_p$.
\begin{itemize}
	\item $a_p=0$, $i_p=1$: we have
  \begin{equation*}
    J = \BIN & \ddots & & \\
             & b_{2p-2} & D_{2p-2} & b_{2p-2}' \BOUT, \quad
    N_\mathcal{A} = \BIN \ddots & \\ & 1 \\ & 1_{a_{p-1}-1} \\ & 1 \BOUT
  \end{equation*}
  the last $a_p-1$ rows of $J_\mathcal{A}$ are $0$.
  \item $a_p = 0$, $i_p=2$:
  \begin{equation*}
    J = \BIN & \ddots   &             &                &              \\
             & b_{2p-2} & D_{2p-2}    & b_{2p-2}'      &              \\
             &          & b_{2p-2}'^T & \delta_{2p-2}' & \beta_{2p-1} \BOUT, \quad
    N_\mathcal{A} = \BIN \ddots & & \\ & 1 & \\ & 1_{a_{p-1}-1} & \\ & 1 & \\ & & 1 \BOUT
  \end{equation*}
  and
  \begin{equation*}
    J_\mathcal{A} = \BIN 0_{a_{p-1}-1,1} & \\ b_{2p-2}'^T 1 + \delta_{2p-2}' & \beta_{2p-1} \BOUT
  \end{equation*}
  \item $a_p = 0$, $i_p>2$:
  \begin{align*}
    &J = \BIN & \ddots   &             &                &            &           \\
             & b_{2p-2} & D_{2p-2}    & b_{2p-2}'      &            &           \\
             &          & b_{2p-2}'^T & \delta_{2p-2}' & b_{2p-1}^T &           \\
             &          &             & b_{2p-1}       & D_{2p-1}   & b_{2p-1}' \BOUT,\\
    &N_\mathcal{A} = \BIN \ddots & & & \\ & 1 & & \\ & 1_{a_{p-1}-1} & & \\ & 1 & & \\ & & I_{i_p-2} & \\ & & & 1 \BOUT
  \end{align*}
  and
  \begin{equation*}
    J_\mathcal{A} = \BIN 0_{a_{p-1}-1,1} & & \\ b_{2p-2}'^T 1 + \delta_{2p-2}' & b_{2p-1}^T \\ b_{2p-1} & D_{2p-1} & b_{2p-1}' \\ \BOUT
  \end{equation*}
  \item $a_p = 1$, $i_p=1$:
  \begin{equation*}
    J = \BIN & \ddots   &             &                &              \\
             & b_{2p-2} & D_{2p-2}    & b_{2p-2}'      &              \\
             &          & b_{2p-2}'^T & \delta_{2p-2}' & \beta_{2p-1} \BOUT, \quad
    N_\mathcal{A} = \BIN \ddots & \\ & 1 \\ & 1_{a_{p-1}-1} \\ & 1 \\ & 0 \BOUT
  \end{equation*}
  and
  \begin{equation*}
    J_\mathcal{A} = \BIN 0_{a_{p-1}-1,1} \\ b_{2p-2}'^T 1 + \delta_{2p-2}' \BOUT
  \end{equation*}
  \item $a_p = 1$, $i_p=2$:
  \begin{align*}
    &J = \BIN & \ddots   &             &                &               &               \\
              & b_{2p-2} & D_{2p-2}    & b_{2p-2}'      &               &               \\
              &          & b_{2p-2}'^T & \delta_{2p-2}' & \beta_{2p-1}  &               \\
              &          &             & \beta_{2p-1}   & \delta_{2p-1} & \beta_{2p-1}' \BOUT,\\
    &N_\mathcal{A} = \BIN \ddots & & \\ & 1 & \\ & 1_{a_{p-1}-1} & \\ & 1 & \\ & & 1 \\ &  & 0 \BOUT
  \end{align*}
  and
  \begin{equation*}
    J_\mathcal{A} = \BIN 0_{a_{p-1}-1,1} & \\ b_{2p-2}'^T 1 + \delta_{2p-2}' & \beta_{2p-1} \\ \beta_{2p-1} & \delta_{2p-1} \\ \BOUT
  \end{equation*}
  \item $a_p = 1$, $i_p>2$:
  \begin{align*}
    &J = \BIN & \ddots   &             &                &             &             &            \\
              & b_{2p-2} & D_{2p-2}    & b_{2p-2}'      &             &             &            \\
              &          & b_{2p-2}'^T & \delta_{2p-2}' & b_{2p-1}^T  &             &            \\
              &          &             & b_{2p-1}       & D_{2p-1}    & b_{2p-1}'   &            \\
              &          &             &                & b_{2p-1}'^T & \delta_{2p} & \beta_{2p} \BOUT,\\
    &N_\mathcal{A} = \BIN \ddots & & & \\ & 1 & & \\ & 1_{a_{p-1}-1} & & \\ & 1 & & \\ & & I_{i_p-2} & \\ & & & 1\\ & & & 0 \BOUT
  \end{align*}
  and
  \begin{equation*}
    J_\mathcal{A} = \BIN 0_{a_{p-1}-1,1}                &             &             \\ 
                         b_{2p-2}'^T 1 + \delta_{2p-2}' & b_{2p-1}^T  &             \\ 
                         b_{2p-1}                       & D_{2p-1}    & b_{2p-1}'   \\ 
                                                        & b_{2p-1}'^T & \delta_{2p} \BOUT
  \end{equation*}
  \item $a_p > 1$, $i_p=1$:
  \begin{equation*}
    J = \BIN & \ddots   &             &                &          &         \\
             & b_{2p-2} & D_{2p-2}    & b_{2p-2}'      &          &         \\
             &          & b_{2p-2}'^T & \delta_{2p-2}' & b_{2p}^T &         \\ 
             &          &             & b_{2p}         & D_{2p}   & \beta_{2p}' \BOUT, \quad
    N_\mathcal{A} = \BIN \ddots & \\ & 1 \\ & 1_{a_{p-1}-1} \\ & 1 \\ & 0_{a_p-1,1} \\ & 0 \BOUT
  \end{equation*}
  and
  \begin{equation*}
    J_\mathcal{A} = \BIN 0_{a_{p-1}-1,1} \\ b_{2p-2}'^T 1 + \delta_{2p-2}' \\  b_{2p} \BOUT
  \end{equation*}
  \item $a_p > 1$, $i_p=2$:
  \begin{align*}
    &J = \BIN & \ddots   &             &                &               &          & \\
              & b_{2p-2} & D_{2p-2}    & b_{2p-2}'      &               &          & \\
              &          & b_{2p-2}'^T & \delta_{2p-2}' & \beta_{2p-1}  &          & \\
              &          &             & \beta_{2p-1}   & \delta_{2p-1} & b_{2p}^T & \\
              &          &             &                & b_{2p}        & D_{2p}   & \beta_{2p}' \BOUT,\\
    &N_\mathcal{A} = \BIN \ddots & & \\ & 1 & \\ & 1_{a_{p-1}-1} & \\ & 1 & \\ & & 1 \\ &  & 0_{a_p-1,1} \\ & & 0\BOUT
  \end{align*}
  and
  \begin{equation*}
    J_\mathcal{A} = \BIN 0_{a_{p-1}-1,1} & \\ b_{2p-2}'^T 1 + \delta_{2p-2}' & \beta_{2p-1} \\ \beta_{2p-1} & \delta_{2p-1} \\ & b_{2p} \\ \BOUT
  \end{equation*}
  \item $a_p > 1$, $i_p>2$:
  \begin{align*}
    &J = \BIN & \ddots   &             &                &             &             &          &         \\
              & b_{2p-2} & D_{2p-2}    & b_{2p-2}'      &             &             &          &         \\
              &          & b_{2p-2}'^T & \delta_{2p-2}' & b_{2p-1}^T  &             &          &         \\
              &          &             & b_{2p-1}       & D_{2p-1}    & b_{2p-1}'   &          &         \\
              &          &             &                & b_{2p-1}'^T & \delta_{2p} & b_{2p}^T &         \\
              &          &             &                &             & b_{2p}      & D_{2p}    & b_{2p}' \BOUT,\\
    &N_\mathcal{A} = \BIN \ddots & & & \\ & 1 & & \\ & 1_{a_{p-1}-1} & & \\ & 1 & & \\ & & I_{i_p-2} & \\ & & & 1\\ & & & 0_{a_p-1,1} \\ & & & 0 \BOUT
  \end{align*}
  and
  \begin{equation*}
    J_\mathcal{A} = \BIN 0_{a_{p-1}-1,1}                &             &             \\ 
                         b_{2p-2}'^T 1 + \delta_{2p-2}' & b_{2p-1}^T  &             \\ 
                         b_{2p-1}                       & D_{2p-1}    & b_{2p-1}'   \\ 
                                                        & b_{2p-1}'^T & \delta_{2p} \\
                                                        &             & b_{2p}      \BOUT
  \end{equation*}
\end{itemize}


\subsection{QR decomposition of $J_\mathcal{A}$}
In this section, we explain how to get cheaply the QR decomposition of $J_\mathcal{A}$. To do so, an intermediate step is to obtain the QR decompositions of the matrices $J_0$, $J_j$ and $J_{p-1}$ that we now describe a bit more in detail. For that we denote by $\BIN e_0,\ldots,e_k\BOUT$ a subpart $\BIN d_{j_0},\ldots,d_{j_0+k}\BOUT$ of the vector $d$ defining $J$. For each description, $j_0$ and $k$ are different and judiciously chosen.

We have 
\begin{equation*}
  J_0 = \BIN
    0_{a_0-1,1} &          &         &                &      \\
    e_0         &          &         &                &      \\
   -e_0-e_1     &  e_1     &         &                &      \\
    e_1         & -e_1-e_2 & e_2     &                &      \\
                &          & \ddots  &                &      \\
                &          & e_{k-1} & -e_{k-1} - e_k &  e_k \\
                &          &         &     e_k        & -e_k
  \BOUT
\end{equation*}
\begin{equation*}
  J_j = \BIN
   -e_0     &  e_0     &          &          &                &      \\
    e_0     & -e_0-e_1 & e_1      &          &                &      \\
            & e_1      & -e_1-e_2 & e_2      &                &      \\
            &          &          & \ddots   &                &      \\
            &          &          & e_{k-1}  & -e_{k-1} - e_k &  e_k \\
            &          &          &          &     e_k        & -e_k
  \BOUT
\end{equation*}
Note that when $i_{2j}$ is $1$ (and thus $I_{i_{2j}-1}$ is empty), $J_j$ reduces to $\BIN -e_0 & e_0 \\ e_0 & -e_0\BOUT$. Furthermore, $J_j$ is $k+2$ by $k+2$, but has rank $k+1$ (the sum of the rows is $0$). Finally,
\begin{equation*}
  J_{p-1} = \BIN
   -e_0     &  e_0     &          &          &                    &              \\
    e_0     & -e_0-e_1 & e_1      &          &                    &              \\
            & e_1      & -e_1-e_2 & e_2      &                    &              \\
            &          &          & \ddots   &                    &              \\
            &          &          & e_{k-2}  & -e_{k-2} - e_{k-1} & e_{k-1}      \\
            &          &          &          &     e_{k-1}        & -e_{k-1}-e_k 
  \BOUT
\end{equation*}
For $J_0$, we will only look for a decomposition
\begin{equation*}
  J_0 = \BIN I & 0 \\ 0 & Q_0 \BOUT \BIN 0_{a_0-1,k+1} \\ R_0 \BOUT
\end{equation*}
apply the QR decomposition to the submatrix starting at $e_0$. This submatrix is a band matrix lower bandwidth $2$ and upper bandwidth $0$ and can be decomposed efficiently using $2k-1$ Givens rotations.

$J_j$ is Hessenberg and tridiagonal. The QR can be obtained with $k+1$ specific Givens rotations. More precisely, let's define
\begin{equation}
G_i = \BIN
  I_{i-1} & & & \\
    & \frac{\sqrt{i+1}}{i+1} & \frac{\sqrt{i}\sqrt{i+1}}{i+1} & \\
    & -\frac{\sqrt{i}\sqrt{i+1}}{i+1} & \frac{\sqrt{i+1}}{i+1} & \\
    & & & I_{k-i+1}
\BOUT
\end{equation}
It can be verified that the QR decomposition is obtained for $Q_j = G_1 G_2 \ldots G_{k+1}$.
\end{document}
